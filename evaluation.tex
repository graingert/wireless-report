\section{Evaluation}


\subsection{Calssifier}

The classification algorithm can detect unknown URLs with the aid of a pattern matching algorithm. The result of the classifier will grade the maliciousness of the given URL with a confidence rate. However, the pattern-matching algorithm is not very fast. The main reason is for any given URL, the classifier has to check the pattern of all URLs in the database to find if there are any matches. After running the system each time, five malware scanners update the database with their returned result. Therefore, the list of URLs in the database increases after each update. As a result, over time the classifier has to undertake more pattern matches for a given unknown URL, which in turn results in decreasing processing speed over time.

\paragraph{} 
According to the many studies performed on the subject of malware detection, applying machine learning algorithms when implementing a classifier will result in an efficient detection system. Employing sophisticated machine learning techniques requires a dataset in advance to enable machine learning to occur for the purpose of classification. However, the latter requirement cannot be fulfilled because this project is time limited.


\subsection{HTML Malware Scanner}

The design of HTML malware scanner is based on the frequency of reappearing trendy-terms within the web page of a given URL. Also, the scanner searches all associated webpage hyperlinked within the given URL. The main advantage of the HTML malware scanner is definition independency, which means the design is independent of the definition of a malicious web page. Also the employed method observed the relation between using trendy-term in the search engine and finding the malicious web page.


\subsection{Evaluation of ClamAV Malware Scan}
Anti-virus scanning gives the system the ability to detect malicious 
signature patterns in various file types that may exist in webpages. The 
testing results proved the fast response and scanning speed of ClamAV. The 
program returns an list of malware found in a particular website and 
is self-contained.

A limitation 
of this project is the exclusion of links to other domains. It might be too 
decisive which is highly possible to miss malicious files, as it is 
not rare to see a webpage contain a direct download link which points to a 
virus executable hosted on another website that cannot be easily found by 
a search engine. Apart from other disadvantages inherited from a low interaction 
detection method, the system does not filter obvious 
harmless file types, such as images and videos. It could have been a more polished system if 
the project schedule was more flexible.



\subsection{Evaluation of Wine Explorer}
Wine Explorer is indeed a feasible approach for high interaction malware 
detection. As Wine is a compatibility layer instead of virtual machine or 
emulator, it enables IE to execute with minimal resource consumption. 
An application run by Wine is also 
treated as ``first class citizens''\cite{wineperformance} and 
permissions for it are not 
restricted. The creation of a Wine prefix is also impressively fast, which 
means low overhead for honeypot resets. 
\paragraph{}
However disadvantages still exists. We explained that the Wine prefix is treated
as a sandbox, and the reason is almost all Windows applications do not operate 
with Linux environment. Imagine if attackers predicted the Wine Explorer 
environment, then they could easily develop Windows applications which inject 
Linux shell commands, which means, they will have access to the file system 
outside of Wine prefix. Also, the suspicious behaviour detection functions of 
this cannot cover all situations. An example 
could 
be a downloaded virus that is able to delete itself before the scan begins. 
During that time it can perform illegal behaviours such as 
sensitive information collection. Modification of the system 
registry is also not considered in this application. 


\subsection{Capture-HPC}
Although Capture-HPC is a very capable malware scanning component, there were
numerous difficulties involved setting it up. Compiling the Windows kernel
drivers for the client was a very complex process, and will need to be repeated
to generate installers for other Windows OSs. The ECS specific modifications
that required two sets of RPC over Message Queues to work also caused considerable
delay to the integration of Capture-HPC into the framework, but means that some
level of network security is maintained separating the hosts running the rest of
the framework from the host running Capture-HPC.

\subsection{Malware Lists}
Although very fast in terms of URLs scanned per second, the components of the system using Malware Lists give very limited results about those URLs. This is not a severe problem as the intention of these components is to avoid scanning high traffic, or high trustworthy sites such as Google or Wikipedia.

\paragraph{}
Difficulties involved with interacting with the counter-intuitive, and in some cases defective, Celery Batches API resulted in more time expended compared to that originally expected. Fortunately it was possible to fix the issues resulting in a defective Batches API and submit those upstream.

\subsection{Framework}
The distributed framework allowed the system to control multiple independent malware scanning sub-systems.  The distributed framework is the best solution for the project because multiple systems can be scanning concurrently on a vast array of hardware. Celery allowed the framework to define Tasks and describe the way in which data must flow through the system: stating which Tasks depended on which other Tasks without directly exposing concepts of Queues and Locks.

\paragraph{}
However while the ability to distribute the system across multiple machines offered a significant performance benefit it also reduced the rate of development and caused difficulties during integration because multi-threaded programming concepts are generally counter-intuitive and difficult to reason about for those not used to the concept.


